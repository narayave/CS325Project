\documentclass[11pt,letterpaper]{article}
\usepackage{anysize}
\usepackage{indentfirst}
\usepackage{sectsty}
\usepackage{amsmath}
\usepackage{hyperref}
\usepackage{graphicx}
\usepackage{chngpage}
\usepackage{enumerate}
\hypersetup{
	colorlinks=true, 
	linkcolor=blue, 
	urlcolor=blue, 
	pdfnewwindow=true, 
	citecolor=black
}
\urlstyle{same}
\linespread{1.2}

\begin{document}

\begin{titlepage}
    \vspace*{4cm}
    \begin{flushright}
    {\huge
        Project 2\\[5mm]
    }
    {\large
        CS325 | Spring 2015
     }
    \end{flushright}
\hrule
    \begin{flushright}
	by Group 2\\
	Vedanth Narayanan\\
	Jonathan Merrill\\
	Tracie Lee\\
    \vfill
	\today\\
    \end{flushright}
\end{titlepage}

\raggedright

\section*{Dynamic Programming Table}

\section*{Algorithm Pseudocode}

\section*{Dynamic Programming Induction Proof}

\section*{Questions}
\begin{enumerate}
	\item Suppose V = [1, 5, 10, 25, 50]. For each integer value of A in [2010, 2015, 2020, …, 2200] determine the number of coins that changegreedy and changedp requires. You can attempt to run changeslow however if it takes too long you can select smaller values of A and also run the other algorithms on the values. Plot the number of coins as a function of A for each algorithm. How do the approaches compare?
	\item Suppose V1 = [1, 2, 6, 12, 24, 48, 60] and V2 = [1, 6, 13, 37, 150]. For each integer value of A in [2000, 2001, 2002, …, 2200] determine the number of coins that changegreedy and changedp requires. If your algorithms run too fast try [10000, 10001, 10003, …, 10100]. You can attempt to run changeslow however if it takes too long you can select smaller values of A and also run all three algorithms on the values. Plot the number of coins as a function of A for each algorithm. How do the approaches compare?
	\item Suppose V = [1, 2, 4, 6, 8, 10, 12, …, 30]. For each integer value of A in [2000, 2001, 2002, …, 2200] determine the number of coins that changegreedy and changedp requires. You can attempt to run changeslow however if it takes too long you can select smaller values of A and also run all three algorithms on the values. Plot the number of coins as a function of A for each algorithm.
	\item For the above situations, determine (experimentally) the running times of the algorithms by fitting trend lines to the data or analyzing the log-log plot. Graph the running time as a function of A. Compare the running times of the different algorithms.
	\item Use the data from questions 4-6 and any new data you have generated. Plot running times as a function of number of denominations (i.e. V=[1, 10, 25, 50] has four different denominations so n=4). Does the size of n influence the running times of any of the algorithms?
	\item Suppose you are living in a country where coins have values that are powers of p, V = [$p^0$, $p^1$, $p^2$, ... , $p^n$]. How do you think the dynamic programming and greedy approaches would compare? Explain.
\end{enumerate}

\end{document}
