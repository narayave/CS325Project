\documentclass[11pt,letterpaper]{article}
\usepackage{anysize}
\usepackage{indentfirst}
\usepackage{sectsty}
\usepackage{amsmath}
\usepackage{hyperref}
\usepackage{graphicx}
\usepackage{chngpage}
\usepackage{enumerate}
\hypersetup{
	colorlinks=true, 
	linkcolor=blue, 
	urlcolor=blue, 
	pdfnewwindow=true, 
	citecolor=black
}
\urlstyle{same}
\linespread{1.2}

\begin{document}

\begin{titlepage}
    \vspace*{4cm}
    \begin{flushright}
    {\huge
        Project 1\\[5mm]
    }
    {\large
        CS325 | Spring 2015
     }
    \end{flushright}
\hrule
    \begin{flushright}
	by Group 2\\
	Vedanth Narayanan\\
	Jonathan Merrill\\
	Tracie Lee\\
    \vfill
	\today\\
    \end{flushright}
\end{titlepage}

\raggedright

\section{Theoretical Run-time Analysis}

\subsection{Algorithm 1}
\begin{verbatim}
%%  ENTER PSEUDO-CODE HERE %%
\end{verbatim}
\textbf{Asymptotic Analysis}\\
We have $O(n^2)$ pairs * $O(n)$ time to compute each sum $= O(n^3)$.

\subsection{Algorithm 2}
\begin{verbatim}
%%  ENTER PSEUDO-CODE HERE %%
\end{verbatim}
\textbf{Asymptotic Analysis}\\
We have $O(n)$ i-iterations (outer loop) * $O(n)$ j-iterations (inner loop) * $O(n)$ for the time to update $= O(n^2)$.

\subsection{Algorithm 3}
\begin{verbatim}
%%  ENTER PSEUDO-CODE HERE %%
\end{verbatim}
\textbf{Asymptotic Analysis}\\
We have $T(n) = 2T(\frac{n}{2}) + \Theta(n)$. This falls within Case 2 of the Master Method, and therefore yields a solution of $\Theta(nlgn)$.

\subsection{Algorithm 4}
\begin{verbatim}
%%  ENTER PSEUDO-CODE HERE %%
\end{verbatim}
\textbf{Asymptotic Analysis}\\
We have $O(n)$ things to compute, therefore this takes $O(n)$ time.


\section{Proof of Correctness: Algorithm 3}
\subsection*{Base Case}
We pass in either an empty array or an array consisting of 1 element. In the first case, an empty array is returned and in the second case the algorithm returns the same array that had been passed in since it is the max subarray within that array.

\subsection*{Inductive Hypothesis}
Assume that algorithm 3 correctly returns a max subarray from an array of n+1 elements.

\subsection*{Inductive Step}

\subsection*{Termination}

\section{Testing}


\section{Experimental Analysis}


\section{Extrapolation and interpretation}


\end{document}
